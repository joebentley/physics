\documentclass[11pt]{amsart}
\usepackage{amsmath,amsfonts,amsthm,amssymb, amsaddr, verbatimbox}


\title{Schr\"{o}dinger equation in 3-D}

\author{Joe Bentley}

\date{\today}

\begin{document}

\maketitle

\newpage

\section{Introduction}

In this we will look at how use the Schr\"{o}dinger equation in three dimensions, as well as seeing how it relates to the one dimensional Schr\"{o}dinger equation. This should be quite simple and does use some of the vector calculus stuff from the mathematics module, but mainly this should be simple to follow for the most part.

\section{Momentum in 3-D and the Schr\"{o}dinger Equation}

Before our concept of momentum was limited to one dimension, which meant we could represent the momentum operator using a single differential operator of the form $\hat{p} = -i\hbar\frac{\partial}{\partial x}$. However now, our momentum has three perpendicular components in the $x$, $y$, and $z$ directions, which are all independent of each other. We can therefore write that we have three momentum operators, one for each momentum eigenvalue corresponding to a given direction,

\begin{align*}
  \hat{p_x} = -i\hbar\frac{\partial}{\partial x} \qquad \hat{p_y} = -i\hbar\frac{\partial}{\partial y} \qquad \hat{p_z} = -i\hbar\frac{\partial}{\partial z}
\end{align*}

We can write this concisely using vector notation as $\hat{p} = (\hat{p_x}, \hat{p_y}, \hat{p_z})$. Substituting each one of these in,

\begin{align*}
  \hat{p}=-i\hbar
  \begin{pmatrix}
    \frac{\partial}{\partial x} \\[4pt]
    \frac{\partial}{\partial y} \\[4pt]
    \frac{\partial}{\partial z}
  \end{pmatrix}
\end{align*}

From vector calculus, the vector here is just the gradient operator, so this can be very concisely written as,

\begin{align*}
  \hat{p}=-i\hbar\nabla
\end{align*}

Similarly, for the kinetic energy operator,

\begin{align*}
  \hat{T}=\frac{{\hat{p}}^2}{2m}=-\frac{\hbar^2}{2m}\nabla^2
\end{align*}

Here the operator $\nabla^2$ is just the Laplacian. We can therefore see that the Schr\"{o}dinger equation, when expanded out of operator form, can be expressed as,

\begin{align*}
  -\frac{\hbar^2}{2m}\nabla^2\Psi+V\Psi=i\hbar\frac{\partial}{\partial t}\Psi
\end{align*}

where $\Psi = \Psi(\mathbf{r}, t)$, and $V = V(\mathbf{r}, t)$; both are a function of three dimensional space ($\mathbf{r} = (x, y, z)$) and time.

If the potential $V$ is only spatially varying, and is constant in time, we can write the time independent Schr\"{o}dinger equation as,

\begin{align*}
  -\frac{\hbar^2}{2m}\nabla^2\psi(\mathbf{r})+V(\mathbf{r})\psi(\mathbf{r})=E\psi(\mathbf{r})
\end{align*}

where, just as before, we say that the full time dependent wavefunction is the product of a spatial and temporal part $\Psi(\mathbf{r},t)=\psi(\mathbf{r})e^{-\frac{iEt}{\hbar}}$.

For a travelling wave in three-dimensions we can extend our one dimensional form by considering the wavefunction as a product of three spatial parts,

\begin{align*}
  \psi(\mathbf{r})&=e^{ik_xx}e^{ik_yy}e^{ik_zz} \\
                  &=e^{i(k_xx+k_yy+k_zz)} \\
                  &=e^{i(\mathbf{k}\cdot\mathbf{r})}
\end{align*}

where the vector $\mathbf{k}=(k_x,k_y,k_z)$ is called the wavevector, and is the vector analogue of the wavenumber.

As before, the wavefunction $\psi(\mathbf{r})$ is an eigenfunction of $\hat{p}$ with an eigenvalue of the three-dimensional momentum,

\begin{align*}
  \hat{p}e^{i\mathbf{k}\cdot\mathbf{r}}&=-i\hbar\nabla e^{i\mathbf{k}\cdot\mathbf{r}}\\
                                       &=\hbar\mathbf{k} e^{i\mathbf{k}\cdot\mathbf{r}}\\
                                       &=\mathbf{p}e^{i\mathbf{k}\cdot\mathbf{r}}\\
\end{align*}


We can write uncertainty relations for each orthogonal direction just as we could before, for example in the $y$-direction we can write $[\hat{y},\hat{p_y}]=i\hbar$ such that we can write our usual relation for the Heisenberg uncertainty principle, $\Delta y\Delta p_y \geq \hbar/2$. This is exactly the same in the $x$ and $y$ directions. However, note that if the observables are perpendicular to each other, such as the $x$-position and the $y$-momentum, there exists no uncertainty relation, as $[\hat{x},\hat{p_y}]=0$ which can be shown easily. This means that we can know both the momentum in one direction and the position in an orthogonal direction precisely and simultaneously.

\section{Solving the Schr\"{o}dinger Equation in 3-D: Infinite Potential Box}

Consider a particle confined to a cuboid potential, with sides length $a$, $b$, $c$, such that,

\begin{align*}
  V(\mathbf{r})=
  \begin{cases}
    0 & 0\leq x\leq a \\
    \dots & 0\leq y\leq b \\
    \dots & 0\leq z\leq c \\
    \infty & \text{elsewhere}
  \end{cases}
\end{align*}

This is perfectly analogous to the one dimensional infinite square well, just extended again into three independent directions. Outside the box the potential is infinite so there is no chance the particle will be there, and thus the wavefunction outside will be zero. Inside however the potential is zero, so the total energy will just be equal to the kinetic energy. Therefore by applying the kinetic energy operator $\hat{T}$,

\begin{align*}
  -\frac{\hbar^2}{2m}\nabla^2\psi(\mathbf{r}) &= -\frac{\hbar^2}{2m}\left(\frac{\partial^2}{\partial x^2} + \frac{\partial^2}{\partial y^2} + \frac{\partial^2}{\partial z^2}\right)\psi(\mathbf{r}) \\
                                              &= E\psi(\mathbf{r})
\end{align*}

There are no cross-dimensional terms as the energy is independent in each orthogonal direction, so we can separate the wavefunction in to a part for each direction,

\begin{align*}
  \psi(\mathbf{r})=X(x)Y(y)Z(z)
\end{align*}

and therefore we have three equations to solve of the form

\begin{align*}
  -\frac{\hbar^2}{2m} \frac{d^2 X(x)}{dx^2} = E_x X(x) \qquad 0\leq x\leq a
\end{align*}

The other two equation are the same but with the $y$ and $z$ coordinates instead. The total energy is written as $E=E_x+E_y+E_z$.

We already know the solutions as they are the same as an infinite square well in one dimension for each direction,

\begin{align*}
  X(x) = A_x\sin{\left(n_x\frac{\pi x}{a}\right)} \qquad n_x\in\mathbb{N}
\end{align*}

and similarly for the $y$ and $z$ directions. Note that the quantum number, $n_x$, can be different for each direction. There is no limitation that requires the quantum number to be the same in every direction! Note that this time we only have sine solutions, and we don't have solutions that switch between sine and cosine. This is because this our potential isn't symmetric about the origin, as one corner of the box is at the origin.

The full solution is then given by,

\begin{align*}
  \psi(\mathbf{r}) = A\sin{\left(\frac{n_x \pi}{a} x\right)}\sin{\left(\frac{n_y \pi}{b} y\right)}\sin{\left(\frac{n_z \pi}{c} z\right)}
\end{align*}

The coefficient of the coordinate in each sine is then just the wavenumber, for example $k_x = \frac{n_x \pi}{a}$. We can find the energy eigenvalues by applying the kinetic energy operator (the potential is zero inside the well) finding,

\begin{align*}
  E=E_x+E_y+E_z=\frac{\pi^2\hbar^2}{2m}\left(\frac{n_x^2}{a^2}+\frac{n_y^2}{b^2}+\frac{n_z^2}{c^2}\right)
\end{align*}

where each of the quantum numbers $n_x, n_y, n_z$ can vary independently.


\section{Solving the Schr\"{o}dinger Equation in 3-D: Infinite Potential Cube}

Consider the special case where the sides are all the same lengths, such that $a = b = c$, and we therefore have a symmetric potential box. Note how now, different combinations of $n_x$, $n_y$, $n_z$ will give us the same energy eigenvalues. This is called degeneracy. We will explore how this degeneracy occurs for each excited state,

\addvbuffer[12pt 8pt]{\begin{tabular}{l c c c}
  State & $(n_x, n_y, n_z)$ & $E$ & Degeneracy value \\[6pt]
  Ground state & $(1, 1, 1)$ & $\frac{3\pi^2\hbar^2}{2ma^2}$ & $1$ \\[6pt]
  First excited state & $(2, 1, 1)$ & $\frac{6\pi^2\hbar^2}{2ma^2}$ & $3$ \\
  & $(1, 2, 1)$ & \\
  & $(1, 1, 2)$ & \\[6pt]
  Second excited state & $(2, 2, 1)$ & $\frac{9\pi^2\hbar^2}{2ma^2}$ & $3$ \\
  & $(2, 1, 2)$ & \\ 
  & $(1, 2, 2)$ & \\[6pt]
  Third excited state & $(2, 2, 2)$ & $\frac{12\pi^2\hbar^2}{2ma^2}$ & $1$
\end{tabular}}

Note that the number of eigenstates with the same energy is called the degeneracy value, and that this degeneracy is due to the symmetry of the potential.

\section{Angular Momentum}

We have not come across angular momentum before as it is a phenomenon which only appears in three dimensions, so we will not have seen it in one-dimension. Classically we define the angular momentum as the cross product of the lever arm distance with the linear momentum,

\begin{align*}
  \mathbf{L} = \mathbf{r}\times\mathbf{p}
\end{align*}

or written in determinant form,

\begin{align*}
  \mathbf{L} =
  \begin{vmatrix}
    i & j & k \\
    x & y & z \\
    p_x & p_y & p_z
  \end{vmatrix}
\end{align*}

Therefore the components of the angular momentum in each Cartesian direction are given by,

\begin{align*}
  L_x &= yp_z - zp_y \\
  L_y &= zp_x - xp_z \\
  L_z &= xp_y - yp_x
\end{align*}

This can be written in operator form using the momentum operator $\hat{p_x} = -i\hbar\frac{\partial}{\partial x}$,

\begin{align*}
  \hat{L_x} &= -i\hbar\left(y\frac{\partial}{\partial z}-z\frac{\partial}{\partial y}\right) \\
  \hat{L_y} &= -i\hbar\left(z\frac{\partial}{\partial x}-x\frac{\partial}{\partial z}\right) \\
  \hat{L_z} &= -i\hbar\left(x\frac{\partial}{\partial y}-y\frac{\partial}{\partial x}\right)
\end{align*}

Now that we know what form our angular momentum operators take, the next equation is, how well can we measure each component simultaneously? To know this we find the commutator,

\begin{align*}
  [\hat{L_x}, \hat{L_y}] &= \hat{L_x}\hat{L_y} - \hat{L_y}\hat{L_x} \\
                         &= -\hbar^2\left\{\left(y\frac{\partial}{\partial z}-z\frac{\partial}{\partial y}\right)\left(z\frac{\partial}{\partial x}-x\frac{\partial}{\partial z}\right)-\left(z\frac{\partial}{\partial x}-x\frac{\partial}{\partial z}\right)\left(y\frac{\partial}{\partial z}-z\frac{\partial}{\partial y}\right)\right\} \\
                         &= -\hbar^2\left\{y\frac{\partial}{\partial x}-x\frac{\partial}{\partial y}\right\} \\
                         &= i\hbar\hat{L_z}
\end{align*}

Similarly, for the other components,

\begin{align*}
  [\hat{L_y}, \hat{L_z}] &= i\hbar\hat{L_x} \\
  [\hat{L_z}, \hat{L_x}] &= i\hbar\hat{L_y}
\end{align*}

Therefore, we see that no two components of the angular momentum commute. This shows us that we can only know one component exactly, and therefore we cannot know $\mathbf{L} = (L_x, L_y, L_z)$ precisely, and cannot know which direction $\mathbf{L}$ points in.

What about the magnitude, $\hat{L^2} = \hat{L_x^2} + \hat{L_y^2} + \hat{L_z^2}$? This time we need to work out the commutator given by,

\begin{align*}
  [\hat{L^2}, \hat{L_z}] &= [\hat{L_x^2} + \hat{L_y^2} + \hat{L_z^2}, \hat{L_z}] \\
                         &= [\hat{L_x^2}, \hat{L_z}] + [\hat{L_y^2}, \hat{L_z}] + [\hat{L_z^2}, \hat{L_z}]
\end{align*}

We know that the last term is zero, as the commutator of an operator with itself is clearly guaranteed to be zero.

To find the rest we need a couple of useful commutator relations,

\begin{align*}
  [\hat{A}\hat{B}, \hat{C}] &= \hat{A}\hat{B}\hat{C}-\hat{C}\hat{A}\hat{B} \\
                            &= \hat{A}\hat{B}\hat{C}-\hat{A}\hat{C}\hat{B}+\hat{A}\hat{C}\hat{B}-\hat{C}\hat{A}\hat{B} \\
                            &= \hat{A}[\hat{B},\hat{C}]+[\hat{A},\hat{C}]\hat{B}
\end{align*}

and,

\begin{align*}
  [\hat{A},\hat{B}]=-[\hat{B},\hat{A}]
\end{align*}

Now we can calculate our first term, using these relations,

\begin{align*}
  [\hat{L_x^2}, \hat{L_z}] &= [\hat{L_x}\hat{L_x}, \hat{L_z}] \\
                           &= \hat{L_x}[\hat{L_x}, \hat{L_z}] + [\hat{L_x}, \hat{L_z}]\hat{L_x} \\
                           &= \hat{L_x}(-i\hbar\hat{L_y}) + (-i\hbar\hat{L_y})\hat{L_x} \\
                           &= -i\hbar(\hat{L_x}\hat{L_y} + \hat{L_y}\hat{L_x}) \neq 0
\end{align*}

And the second term, similarly,

\begin{align*}
  [\hat{L_y^2}, \hat{L_z}] &= [\hat{L_y}\hat{L_y}, \hat{L_z}] \\
                           &= \hat{L_y}[\hat{L_y}, \hat{L_z}] + [\hat{L_y}, \hat{L_z}]\hat{L_y} \\
                           &= \hat{L_y}(-i\hbar\hat{L_x}) + (-i\hbar\hat{L_x})\hat{L_y} \\
                           &= i\hbar(\hat{L_y}\hat{L_x} + \hat{L_x}\hat{L_y}) = -[\hat{L_x^2}, \hat{L_z}]
\end{align*}

We therefore find by adding each term together the commutator for the magnitude of the angular momentum and the $z$-component,

\begin{align*}
  [\hat{L^2}, \hat{L_z^2}] &= -i\hbar(\hat{L_x}\hat{L_y}+\hat{L_y}\hat{L_x}) + i\hbar(\hat{L_x}\hat{L_y}+\hat{L_y}\hat{L_x}) + 0 \\
                           &= 0
\end{align*}

Therefore we can know both the magnitude of the angular momentum and the $z$-component simultaneously and precisely. Similarly this is true for the $x$ and $y$-components as well which can be shown by calculating their commutators. Therefore we see that since $\hat{L^2}$ commutes with each of $\hat{L_x}$, $\hat{L_y}$, and $\hat{L_z}$, then $\hat{L^2}$ shares a set of simultaneous eigenfunctions with $\hat{L_x}$, $\hat{L_y}$, and $\hat{L_z}$. However there is a different set of eigenfunctions for each component as $\hat{L_x}$, $\hat{L_y}$, and $\hat{L_z}$ do not commute with each other in any order. Therefore we can know precisely the value of $L^2$ but can only know a single component of $\mathbf{L}$. For simplicity we choose the $L_z$ component as that single component as it has the simplest form in spherical polar coordinates.


\section{Central Potentials}

A central potential is defined such that the potential energy is only dependent on the radial distance from the origin. This is the same as a Coulomb potential due to point-like electric charge. This is much simpler to work out in spherical polar coordinates, so we will transform our potential into spherical polars. Since our potential is just a function of the radial distance, it is just a function of the spherical polar coordinate $r$.

The Cartesian coordinates can be written in terms of the spherical polar coordinates using the following relations,

\begin{align*}
  x &= r\sin{\theta}\cos{\phi} \\
  y &= r\sin{\theta}\sin{\phi} \\
  z &= r\cos{\theta}
\end{align*}

We call $\theta$ the polar angle, and $\phi$ the azimuthal angle.

The $z$-component of the angular momentum can be written in the form (which will be proven),

\begin{align*}
  \hat{L_z} &= -i\hbar\left(x\frac{\partial}{\partial y}-y\frac{\partial}{\partial x}\right) \\
            &= -i\hbar\frac{\partial}{\partial\phi}
\end{align*}

This is because rotation about the axis is not dependent on the polar angle $\theta$.

We can show that this is true by finding $\frac{\partial}{\partial\phi}$ in terms of $\frac{\partial}{\partial x}$, $\frac{\partial}{\partial y}$, $\frac{\partial}{\partial z}$. By using the chain rule we see that,

\begin{align*}
  \frac{\partial}{\partial\phi}=\frac{\partial}{\partial x}\frac{\partial x}{\partial\phi}+\frac{\partial}{\partial y}\frac{\partial y}{\partial\phi}+\frac{\partial}{\partial z}\frac{\partial z}{\partial\phi}
\end{align*}

The last term is clearly zero as the $z$-component has no dependence on the azimuthal angle $\phi$. We now need to calculate each of the partial derivatives,

\begin{alignat*}{2}
  \frac{\partial x}{\partial\phi} &= -r\sin{\theta}\sin{\phi} &&= -y \\
  \frac{\partial y}{\partial\phi} &= r\sin{\theta}\cos{\phi}  &&= x \\
\end{alignat*}

Therefore we see that $\frac{\partial}{\partial\phi}$ is indeed such that,

\begin{align*}
  \hat{L_z}=-i\hbar\left(x\frac{\partial}{\partial y}-y\frac{\partial}{\partial x}\right)=-i\hbar\frac{\partial}{\partial\phi}
\end{align*}

$\hat{L_z}$ has the simplest form in spherical polar coordinates, which is why we chose it earlier when talking about the commutator. For reference here I will also write $\hat{L_x}$ and $\hat{L_y}$ in spherical polar coordinates, but I will not derive them,

\begin{align*}
  \hat{L_x} &= i\hbar\left(\sin{\phi}\frac{\partial}{\partial\theta}+\cot{\theta}\cos{\phi}\frac{\partial}{\partial\phi}\right) \\
  \hat{L_y} &= i\hbar\left(-\cos{\phi}\frac{\partial}{\partial\theta}+\cot{\theta}\sin{\phi}\frac{\partial}{\partial\phi}\right) \\
\end{align*}

and the magnitude is given by,

\begin{align*}
  \hat{L^2} = -\hbar^2\left(\frac{1}{\sin{\theta}}\frac{\partial}{\partial\theta}\left(\sin{\theta}\frac{\partial}{\partial\theta}\right)+\frac{1}{\sin^2{\theta}}\frac{\partial^2}{\partial\phi^2}\right)
\end{align*}

Note that all of them are dependent on both the polar and azimuthal angle except for the $z$-component of the angular momentum which is only dependent on the azimuthal angle. \textit{We are not expected to memorise these formulae.}

\section{Single Electron Atoms}

In this section will we will solve the time independent Schr\"{o}dinger equation for an electron in a central Coulomb potential given by,

\begin{align*}
  V(r) = \frac{q_1 q_2}{4\pi\epsilon_0 r}
\end{align*}

The charge $q_1 = Ze$ is the charge of the nucleus with atomic (proton) number $Z$. The charge $q_2 = -e$ is the charge of a single electron. Since there is only one electron this is a hydrogen-like atom.

Classically the electron and nucleus will rotate about a common centre-of-mass, such that the electron and nucleus are seperated by a distance $r$. However it is easier to assume that the mass of the nucleus goes to infinity. To keep the distance equal to $r$ we then have to decrease the mass of the electron to a reduced mass to shift the centre of mass back to where it was. This new mass is given by,

\begin{align*}
  \mu = \left(\frac{M}{M+m}\right) M
\end{align*}

where $m$ is the mass of the electron, and $M$ is the mass of the nucleus. The problem is now reduced to one particle moving in a Coulomb potential, and thus the time independent Schr\"{o}dinger equation becomes,

\begin{align*}
  \left(-\frac{\hbar^2}{2\mu}\nabla^2+V(r)\right)\psi(r,\theta,\phi)=E\psi(r,\theta,\phi)
\end{align*}


\end{document}
