\documentclass[11pt]{amsart}
\usepackage{amsmath,amsfonts,amsthm,amssymb, amsaddr}


\title{Principles of Quantum Mechanics}

\author{Joe Bentley}

\date{\today}

\begin{document}

\maketitle

\newpage

\section{Mixed States}

If $\psi_n(x)$ is a solution of the time independent Schr\"{o}dinger equation, then $\psi_n(x)$ is an eigenfunction of the Hamiltonian operator with an energy eigenvalue. The full time-dependant form of the wavefunction is given by,

\begin{align*}
  \Psi_n(x, t) = \psi_n(x) e^{-\frac{iE_nt}{\hbar}}
\end{align*}

When we find the probability density, multiply the wavefunction by its complex conjugate. Since the exponential is negative when we take the complex conjugate, they cancel to give,

\begin{align*}
  {|\Psi_n|}^2 = \psi^*_n\psi_n
\end{align*}

Therefore since $\psi_n$ has no time dependence, neither does the probability density ${|\Psi_n|}^2$.

Now we consider a mixed state of wavefunctions,

\begin{align*}
  \Phi(x, t) = \Psi_m(x, t) + \Psi_n(x, t)
\end{align*}

By taking the probability density of this,

\begin{align*}
  {|\Phi|}^2 &= {(\Psi_m + \Psi_n)}^* (\Psi_m + \Psi_n) \\
               &= \Psi_m^* \Psi_m + \Psi_n^* \Psi_n + \Psi_m^* \Psi_n + \Psi_n^* \Psi_m
\end{align*}

The two cross terms at the end of the second line we call the interference terms, which we define as $z$ and $z^*$ such that,

\begin{alignat*}{2}
  z &= \Psi^*_m \Psi_n &&= \psi_m^* \psi_n e^{-i(E_n - E_m)t / \hbar} \\
  z^* &= \Psi^*_m \Psi_n &&= \psi_m^* \psi_n e^{i(E_n - E_m)t / \hbar}
\end{alignat*}

We see that this satisfies $z + z^* = 2 Re\{z\}$. We can therefore write the probability density as,

\begin{align*}
  {|\Phi|}^2 = {|\psi_m|}^2 + {|\psi_n|}^2 + 2\psi_m^* \psi_n \cos{\left(\frac{\Delta E t}{\hbar}\right)}
\end{align*}

where $\Delta E = E_n - E_m$. Also the assumption has been made that $\psi_m^*\psi_n$ is a real quantity. We therefore see that the probability density of a mixed state $\Phi(x, t)$ varies with time unlike previous examples where the probability density function is constant in time. The mixed state $\Phi$ is an example of a superposition of states. Since Schr\"{o}dinger's equation is linear, any linear combination of mixed states is also a solution,

\begin{align*}
  \Phi = C_m \Psi_m + C_n \Psi_n
\end{align*}

Note that the time dependent solutions are used here. A mixed state of two time independent solutions, $\phi(x) = \psi_m(x) + \psi_n(x)$ would not be a solution of the time independent Schr\"{o}dinger equation, as the energy would need to be well defined, which is isn't for a mixed state,

\begin{align*}
  \hat{H}\phi = \hat{H}\psi_m + \hat{H}\psi_n = E_m\psi_m + E_n\Psi_n
\end{align*}

We see that there are two different energies defined, one for each state so this can't be a solution of the time independent Schr\"{o}dinger equation which has a single well defined energy.

\section{Superposition and Measurement}

Let $\Psi_i(x, t)$ where $i = 1, \dots, n$ form a complete set of solutions to the Schr\"{o}dinger equation for a given problem. The general solution is then given by,

\begin{align*}
  \Phi(x, t) = \sum\limits_{i=1}^n c_i \Psi_i(x, t)
\end{align*}

where $c_i$ are complex coefficients. We postulate that \textit{any} valid set of solutions (and thus any valid wavefunction) can be written in this form.

If $\hat{A}\Psi_i = a_i\Psi_i$ then the observable $A$ (which corresponds to the operator $\hat{A}$ and the eigenvalue $a_i$) is well defined for the state. That is, $\Psi_i$ is an eigenstate of the operator $\hat{A}$. The real eigenvalues $a_i$ represent all the possible results of measuring the observable $A$ for any state, which must be real as they correspond to an observable $A$.

The class of operators that always have real eigenvalues are known as Hermitian operators. An operator $\hat{o}$ is Hermitian \textit{only} if,

\begin{align*}
  \int_{-\infty}^{\infty} \Psi_b^*\hat{o}\Psi_a dx = \int_{-\infty}^{\infty} \Psi_a(\hat{o}\Psi_b)^* dx
\end{align*}

where $\Psi_a$ and $\Psi_b$ are any arbitrary function of $(x, t)$. An operator that this is true for is called a self-adjoint operator,

\begin{align*}
  \hat{o} = \hat{o}^{\dagger}
\end{align*}

A Hermitian matrix is similarly defined,

\begin{align*}
  Q &= Q^{\dagger} \\
  Q_{nm} &= Q_{mn}^*
\end{align*}

We can show that because Hermitian operators are self-adjoint, that they are required to be real. Consider an arbitrary operator $\hat{o}$ acting on a wavefunction $\Psi_a$ which is an eigenfunction of the operator $\hat{o}$.

\begin{align*}
  \text{Let} \qquad \hat{o} \Psi_a = a \Psi_a
\end{align*}

If the operator $\hat{o}$ is Hermitian, then,

\begin{align*}
  \int\Psi_a^*\hat{o}\Psi_a dx &= \int\Psi_a(\hat{o}\Psi_a)^* dx \\
  \int\Psi_a^* a \Psi_a dx &= \int\Psi_a(a \Psi_a)^* dx \\
  a \int\Psi_a^* \Psi_a dx &= a^* \int\Psi_a \Psi_a^* dx \\
\end{align*}

Assuming the wavefunction $\Psi_a$ is normalized (but even if it isn't, the normalisation constants will cancel anyway),

\begin{align*}
  \int\Psi_a^*\Psi_a dx = 1
\end{align*}

Therefore we come to the conclusion that the Hermitian operator always gives real eigenvalues,

\begin{align*}
  a = a^*
\end{align*}

We postulate that all operators corresponding to observables are Hermitian.

What about the relationship between two different eigenfunctions when the Hermitian is applied? First define $\Psi_b$ as another eigenfunction of $\hat{o}$, such that,

\begin{align*}
  \hat{o}\Psi_b = b\Psi_b
\end{align*}

The Hermitian operator satisfies,

\begin{align*}
  \int\Psi_b^*\hat{o}\Psi_a dx = \int\Psi_a(\hat{o}\Psi_b)^* dx
\end{align*}

Therefore by applying the operator,

\begin{align*}
  a\int\Psi_b^*\Psi_a dx &= b\int\Psi_a\Psi_b^* dx \\
  \left(a - b\right) \int\Psi_b^*\Psi_a dx &= 0
\end{align*}

For this to be true we can have two cases. Either $a = b$ and the integral is one, which implies that $\Psi_a = \Psi_b$ or just $\Psi_a = \Psi_a$, or $a \neq b$, which requires that the integral is zero. In this case we say that the eigenfunctions $\Psi_a$ and $\Psi_b$ are orthogonal, which we can define succinctly using the Kronecker delta,

\begin{align*}
  \int\Psi_b^*\Psi_a dx = \delta_{ab}
\end{align*}

\section{Expectation Values for Superposition States}

Consider the superposition state,

\begin{align*}
  \Phi = \sum\limits_{i=1}^n c_i \Psi_i
\end{align*}

which is a general solution to the Schr\"{o}dinger equation. We define the operator $\hat{o}$, such that $\Psi_i$ is an eigenfunction, that is, $\hat{o}\Psi_i = a_i\Psi_i$ where $a_i$ is a real eigenvalue corresponding to an observable $o$. The expectation value of $o$ can be written as,

\begin{align*}
  \langle o\rangle = \int_{-\infty}^{\infty} \Phi^* \hat{o} \Phi dx
\end{align*}

Assuming that $\Phi$ is normalized.

We want to show the result of finding the expectation value of a mixed state $\Phi$,

\begin{align*}
  \langle o\rangle &= \int_{-\infty}^{\infty} \left(\sum\limits_i c_i \Psi_i\right)^* \hat{o} \left(\sum\limits_j c_j \Psi_j\right) dx \\
                   &= \int_{-\infty}^{\infty} \sum\limits_i c_i^* \Psi_i^* \sum\limits_j c_j a_j \Psi_j \\
                   &= \sum\limits_i\sum\limits_j c_i^* c_j a_j \int_{-\infty}^{\infty} \Psi_i^*\Psi_j dx \\
                   &= \sum\limits_i\sum\limits_j c_i^* c_j a_j \delta_{ij}
\end{align*}

We know that the integral can be written as the Kronecker delta, because it is normalized, it obeys,

\begin{align*}
  \delta_{ij} =
  \begin{cases}
    0 & i \neq j \\
    1 & i = j
  \end{cases}
\end{align*}

This is because, as we showed before, if $i \neq j$ the wavefunctions $\Psi_i$ and $\Psi_j$ are orthogonal, and thus the integral over them is zero. If $i = j$, due to normalization, the integral over them is one. Therefore the Kronecker delta is satisfied.

Since the Kronecker delta is zero for indices where $i \neq j$ we can rewrite the expectation value as a single sum,

\begin{align*}
  \langle o\rangle &= \sum\limits_j c_j^* c_j a_j \\
  \langle o\rangle &= \sum\limits_j {|c_j|}^2 a_j
\end{align*}

Remember that $c_i$ is the weighting of the wavefunction $\Psi_i$ in the mixed state. The expectation is the weighted mean of all possible values of the eigenvalue and therefore, we can say that ${|c_j|}^2$ is the probability of measuring $a_j$ in any one measurement.

Furthermore, since ${|c_j|}^2$ is the probability, it is implied that $\sum\limits_j {|c_j|}^2 = 1$, which we can show,

\begin{align*}
  \int_{-\infty}^{\infty} \Phi^*\Phi dx &= 1 \\
  \int_{-\infty}^{\infty} \sum\limits_i c_i^* \Psi_i^* \sum\limits_j c_j \Psi_j dx &= 1 \\
  \sum\limits_i \sum\limits_j c_i^* c_j \int_{-\infty}^{\infty} \Psi_i^* \Psi_j dx &= 1 \\
  \sum\limits_i \sum\limits_j c_i^* c_j \delta_{ij} &= 1 \\
  \sum\limits_j {|c_j|}^2 &= 1
\end{align*}

Therefore we have shown,

\begin{align*}
  \int_{-\infty}^{\infty} \Phi^*\Phi dx = \sum\limits_j {|c_j|}^2 = 1
\end{align*}

\section{Example: Expectation Value}

Consider a particle confined in a potential well of width $a$ with a spatial wavefunction given by,

\begin{align*}
  \phi(x) = \sqrt{\frac{2}{a}} \cos{\left(\frac{\pi x}{a}\right)}
\end{align*}

The wavefunction $\phi(x)$ is an eigenstate of the kinetic energy $T$,

\begin{align*}
  \hat{T} = -\frac{\hbar^2}{2m} \frac{d^2}{dx^2} \to E = \frac{\pi^2\hbar^2}{2ma^2}
\end{align*}

However it is not an eigenstate of the momentum operator,

\begin{align*}
  \hat{p} = -i\hbar\frac{d}{dx}
\end{align*}

Eigenfunctions of $\hat{p}$ can be written in the form,

\begin{align*}
  \psi_+(x) &= \frac{1}{\sqrt{a}} e^{\frac{i\pi x}{a}} \\
  \psi_-(x) &= \frac{1}{\sqrt{a}} e^{-\frac{i\pi x}{a}}
\end{align*}

Note that they are normalized with the normalization constants that can be calculated,

\begin{align*}
  A^2 \int_{-a}^{a} \psi_+^* \psi_+ dx = 1 \to A = \frac{1}{\sqrt{a}}
\end{align*}

We can write our superposition $\phi(x)$ in terms of these two eigenfunctions of $\hat{p}$ by rewriting $\cos$ in its exponential form,

\begin{align*}
  \phi(x) = \sqrt{\frac{2}{a}} \left(\frac{e^{\frac{i\pi x}{a}} + e^{-\frac{i\pi x}{a}}}{2}\right)
\end{align*}

By some rearrangement we can write this in terms of our eigenfunctions of $\hat{p}$,

\begin{align*}
  \phi(x) = \frac{1}{\sqrt{2}} \psi_+(x) + \frac{1}{\sqrt{2}} \psi_-(x)
\end{align*}

Writing this in the form $\phi = \sum\limits_i c_i \psi_i$,

\begin{align*}
  \phi(x) = c_+ \psi_+(x) + c_- \psi_-(x)
\end{align*}

The probability of measuring each eigenvalue of momentum is thus given by,

\begin{alignat*}{3}
  p_+ &= &&\frac{\hbar\pi}{a}  &&\to P = {|c_+|}^2 = \frac{1}{2} \\
  p_- &= -&&\frac{\hbar\pi}{a} &&\to P = {|c_-|}^2 = \frac{1}{2}
\end{alignat*}

We can see that in $\phi$ is contained two superimposed eigenfunctions that are eigenfunctions of $\hat{p}$, while $\phi$ itself isn't an eigenstate of $\hat{p}$.

The expectation value of $p$ is given by,

\begin{align*}
  \langle p \rangle = \frac{1}{2}\left(\frac{\hbar\pi}{a}\right) - \frac{1}{2}\left(\frac{\hbar\pi}{a}\right) = 0
\end{align*}

In general for a superposition state of many wavefunctions given by $\phi(x) = \sum\limits_i c_i \psi_i(c)$ we need to use the orthogonality of the wavefunctions to extract the coefficients $c_j$,

\begin{align*}
  \int\psi_j^*\phi dx &= \sum\limits_i c_i \int \psi_j^* \psi_i dx \\
                      &= \sum\limits_i c_i \delta_{ij} \\
                      &= c_j
\end{align*}

We call the integral on the left an overlap integral as we are finding how the eigenfunction $\psi_j^*$ overlaps with the superposition of eigenstates $\phi$.

Using this overlap integral we can calculate the weightings $c_j$ from the previous example, checking that they give us the same answer as before,

\begin{align*}
  c_+ &= \int_{-a/2}^{a/2} \psi^*_+(x) \phi(x) dx \\
      &= \int_{-a/2}^{a/2} {\left(\frac{1}{\sqrt{a}}\right)}^* \left(\sqrt{\frac{2}{a}} \cos{\left(\frac{\pi x}{a}\right)}\right) dx \\
      &= \frac{\sqrt{2}}{a} \int_{-a/2}^{a/2} \left(\cos{\left(\frac{\pi x}{a}\right)} - i\sin{\left(\frac{\pi x}{a}\right)}\right) \cos{\left(\frac{\pi x}{a}\right)} dx \\
      &= \frac{\sqrt{2}}{a} \int_{-a/2}^{a/2} \cos^2{\left(\frac{\pi x}{a}\right)} dx \\
      &= \frac{1}{a\sqrt{2}} \int_{-a/2}^{a/2} \left(1 + \cos{\left(\frac{2\pi x}{a}\right)}\right) dx \\
      &= \frac{1}{\sqrt{2}}
\end{align*}

In the third line we noticed that $\sin(x) \cos(x)$ is zero over symmetric limits, so we know that term will be zero. We also noticed in the second to last line, $\cos(x)$ is zero over symmetric limits, so that also goes.

The weighting is thus given by $c_+ = \frac{1}{\sqrt{2}}$ and is the same as we found before. This is a general method that can be used to find $c_j$ for any eigenfunction in a superposition of eigenstates, as long as $\phi(x)$ and $\psi_j(x)$ are known.

\section{Summary}

Consider a particle in a superposition of eigenstates (that are solutions to the Schr\"{o}dinger equation) as described by,

\begin{align*}
  \Phi(x, t) = \sum\limits_i c_i \Psi_i(x, t)
\end{align*}

The individual states are eigenfunctions of the hamiltonian, such that,

\begin{align*}
  \hat{H} \Psi_i = i\hbar\frac{\partial}{\partial t} \Psi_i
\end{align*}

where $\hat{H} = \frac{\hat{p}^2}{2m} + V(x, t)$.

In cases where the potential $V$ is only dependent on space, and not time, we can write the superposition in the form,

\begin{align*}
  \Phi(x, t) &= \sum\limits_j c_j \psi_j(x) e^{-\frac{iE_j t}{\hbar}} \\
  \phi(x) &= \sum\limits_i c_i \psi_i(x)
\end{align*}

Let $\psi_i(x)$ where $i = 1 \dots n$ be eigenfunctions of the operator $\hat{o}$ such that $\hat{o}\psi_i = a_i \psi_i$. What happens when the observable $o$ is measured? The result will be one of the possible eigenvalues $a_j$, with a probability of measuring the particle in that eigenvalue given by ${|c_j|}^2$. If we measure the observable $o$ again on the same particle, the result will be the same eigenvalue $a_j$, but it will be \textit{certain}. That is, repeated measurements of $o$ will give the same eigenvalue $a_j$ with a probability of unity. This is because the measurement of a eigenvalue is said to collapse the wavefunction from being a superposition of multiple eigenstates to one specific eigenstate. That is, instead of the particle being in a superposition of eigenstates $\phi(x)$ is collapses to one eigenstate $\psi(x)$. The act of measuring the quantity itself causes this collapse.

Measurement of the same observable $o$ on \textit{many} particles will yield different results each time, with the average result given by the expectation value,

\begin{align*}
  \langle o \rangle = \sum\limits_i {|c_i|}^2 a_i
\end{align*}


\section{Simultaneous Eigenfunctions}

A wavefunction can be an eigenfunction of more than one operator at the same time. For example consider the free particle wavefunction,

\begin{align*}
  \Psi(x, t) = Ae^{i(kx - \omega t)}
\end{align*}

This is an eigenfunction of both the energy operator $\hat{E}$ and the momentum operator $\hat{p}$,

\begin{alignat*}{3}
  \hat{p} &= -i\hbar\frac{\partial}{\partial x} \qquad &&\hat{p}\Psi = p\Psi \qquad &&p=\hbar k\\
  \hat{E} &= i\hbar\frac{\partial}{\partial t} \qquad &&\hat{E}\Psi = E\Psi \qquad &&E=\hbar \omega
\end{alignat*}

However it is not generally true that wavefunctions are eigenstates of both $\hat{p}$ and $\hat{E}$. This happens when the operators commute. Commutation is when the order of operations don't matter, for example numbers are commutative under multiplication and addition, that is $xy = yx$ and $y + x = y + x$. Operators however do not usually commute,

\begin{align*}
  x \frac{d}{dx}\left(f(x)\right) \neq \frac{d}{dx}\left(x f(x)\right)
\end{align*}

In quantum mechanics, operators generally don't commute either,

\begin{align*}
  \hat{A}(\hat{B}\psi) &\neq \hat{B}(\hat{A}) \\
  (\hat{A}\hat{B} - \hat{B}\hat{A})\psi &\neq 0
\end{align*}

We call the left hand side of the second line the commutator of $\hat{A}$ and $\hat{B}$, and write it as $[\hat{A}, \hat{B}]$. If $[\hat{A}, \hat{B}] = 0$ the operators are said to commute. Note that this condition is independent of $\psi$, as if $\psi$ is zero then it does not describe the particle at all. This means that we can write,

\begin{align*}
  \text{if } [\hat{A}, \hat{B}] \neq 0 &\quad \text{$\hat{A}$ and $\hat{B}$ do not commute} \\
  \text{if } [\hat{A}, \hat{B}] = 0 &\quad \text{$\hat{A}$ and $\hat{B}$ commute}
\end{align*}

\end{document}
