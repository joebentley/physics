\documentclass[11pt]{amsart}
\usepackage{amsmath,amsfonts,amsthm,amssymb, amsaddr}


\title{Principles of Quantum Mechanics}

\author{Joe Bentley}

\date{\today}

\begin{document}

\maketitle

\newpage

\section{Mixed States}

If $\psi_n(x)$ is a solution of the time independent Schr\"{o}dinger equation, then $\psi_n(x)$ is an eigenfunction of the Hamiltonian operator with an energy eigenvalue. The full time-dependant form of the wavefunction is given by,

\begin{align*}
  \Psi_n(x, t) = \psi_n(x) e^{-\frac{iE_nt}{\hbar}}
\end{align*}

When we find the probability density, multiply the wavefunction by its complex conjugate. Since the exponential is negative when we take the complex conjugate, they cancel to give,

\begin{align*}
  {|\Psi_n|}^2 = \psi^*_n\psi_n
\end{align*}

Therefore since $\psi_n$ has no time dependence, neither does the probability density ${|\Psi_n|}^2$.

Now we consider a mixed state of wavefunctions,

\begin{align*}
  \Phi(x, t) = \Psi_m(x, t) + \Psi_n(x, t)
\end{align*}

By taking the probability density of this,

\begin{align*}
  {\|\Phi\|}^2 &= {(\Psi_m + \Psi_n)}^* (\Psi_m + \Psi_n) \\
               &= \Psi_m^* \Psi_m + \Psi_n^* \Psi_n + \Psi_m^* \Psi_n + \Psi_n^* \Psi_m
\end{align*}

The two cross terms at the end of the second line we call the interference terms, which we define as $z$ and $z^*$ such that,

\begin{alignat*}{2}
  z &= \Psi^*_m \Psi_n &&= \psi_m^* \psi_n e^{-i(E_n - E_m)t / \hbar} \\
  z^* &= \Psi^*_m \Psi_n &&= \psi_m^* \psi_n e^{i(E_n - E_m)t / \hbar}
\end{alignat*}

We see that this satisfies $z + z^* = 2 Re\{z\}$. We can therefore write the probability density as,

\begin{align*}
  {\|\Phi\|}^2 = {\|\psi_m\|}^2 + {\|\psi_n\|}^2 + 2\psi_m^* \psi_n \cos{\left(\frac{\Delta E t}{\hbar}\right)}
\end{align*}

where $\Delta E = E_n - E_m$. Also the assumption has been made that $\psi_m^*\psi_n$ is a real quantity. We therefore see that the probability density of a mixed state $\Phi(x, t)$ varies with time unlike previous examples where the probability density function is constant in time. The mixed state $\Phi$ is an example of a superposition of states. Since Schr\"{o}dinger's equation is linear, any linear combination of mixed states is also a solution,

\begin{align*}
  \Phi = C_m \Psi_m + C_n \Psi_n
\end{align*}

Note that the time dependent solutions are used here. A mixed state of two time independent solutions, $\phi(x) = \psi_m(x) + \psi_n(x)$ would not be a solution of the time independent Schr\"{o}dinger equation, as the energy would need to be well defined, which is isn't for a mixed state,

\begin{align*}
  \hat{H}\phi = \hat{H}\psi_m + \hat{H}\psi_n = E_m\psi_m + E_n\Psi_n
\end{align*}

We see that there are two different energies defined, one for each state so this can't be a solution of the time independent Schr\"{o}dinger equation which has a single well defined energy.

\section{Superposition and Measurement}

Let $\Psi_i(x, t)$ where $i = 1, \dots, n$ form a complete set of solutions to the Schr\"{o}dinger equation for a given problem. The general solution is then given by,

\begin{align*}
  \Phi(x, t) = \sum\limits_{i=1}^n c_i \Psi_i(x, t)
\end{align*}

where $c_i$ are complex coefficients. We postulate that \textit{any} valid set of solutions (and thus any valid wavefunction) can be written in this form.

If $\hat{A}\Psi_i = a_i\Psi_i$ then the observable $A$ (which corresponds to the operator $\hat{A}$ and the eigenvalue $a_i$) is well defined for the state. That is, $\Psi_i$ is an eigenstate of the operator $\hat{A}$. The real eigenvalues $a_i$ represent all the possible results of measuring the observable $A$ for any state, which must be real as they correspond to an observable $A$.

The class of operators that always have real eigenvalues are known as Hermitian operators. An operator $\hat{o}$ is Hermitian \textit{only} if,

\begin{align*}
  \int_{-\infty}^{\infty} \Psi_b^*\hat{o}\Psi_a dx = \int_{-\infty}^{\infty} \Psi_a(\hat{o}\Psi_b)^* dx
\end{align*}

where $\Psi_a$ and $\Psi_b$ are any arbitrary function of $(x, t)$. An operator that this is true for is called a self-adjoint operator,

\begin{align*}
  \hat{o} = \hat{o}^{\dagger}
\end{align*}

A Hermitian matrix is similarly defined,

\begin{align*}
  Q &= Q^{\dagger} \\
  Q_{nm} &= Q_{mn}^*
\end{align*}

We can show that because Hermitian operators are self-adjoint, that they are required to be real. Consider an arbitrary operator $\hat{o}$ acting on a wavefunction $\Psi_a$ which is an eigenfunction of the operator $\hat{o}$.

\begin{align*}
  \text{Let} \qquad \hat{o} \Psi_a = a \Psi_a
\end{align*}

If the operator $\hat{o}$ is Hermitian, then,

\begin{align*}
  \int\Psi_a^*\hat{o}\Psi_a dx &= \int\Psi_a(\hat{o}\Psi_a)^* dx \\
  \int\Psi_a^* a \Psi_a dx &= \int\Psi_a(a \Psi_a)^* dx \\
  a \int\Psi_a^* \Psi_a dx &= a^* \int\Psi_a \Psi_a^* dx \\
\end{align*}

Assuming the wavefunction $\Psi_a$ is normalized (but even if it isn't, the normalisation constants will cancel anyway),

\begin{align*}
  \int\Psi_a^*\Psi_a dx = 1
\end{align*}

Therefore we come to the conclusion that the Hermitian operator always gives real eigenvalues,

\begin{align*}
  a = a^*
\end{align*}

We postulate that all operators corresponding to observables are Hermitian.

What about the relationship between two different eigenfunctions when the Hermitian is applied? First define $\Psi_b$ as another eigenfunction of $\hat{o}$, such that,

\begin{align*}
  \hat{o}\Psi_b = b\Psi_b
\end{align*}

The Hermitian operator satisfies,

\begin{align*}
  \int\Psi_b^*\hat{o}\Psi_a dx = \int\Psi_a(\hat{o}\Psi_b)^* dx
\end{align*}

Therefore by applying the operator,

\begin{align*}
  a\int\Psi_b^*\Psi_a dx &= b\int\Psi_a\Psi_b^* dx \\
  \left(a - b\right) \int\Psi_b^*\Psi_a dx &= 0
\end{align*}

For this to be true we can have two cases. Either $a = b$ and the integral is one, which implies that $\Psi_a = \Psi_b$ or just $\Psi_a = \Psi_a$, or $a \neq b$, which requires that the integral is zero. In this case we say that the eigenfunctions $\Psi_a$ and $\Psi_b$ are orthogonal, which we can define succinctly using the Kronecker delta,

\begin{align*}
  \int\Psi_b^*\Psi_a dx = \delta_{ab}
\end{align*}



\end{document}
