\documentclass[11pt]{amsart}
\usepackage{amsmath,amsfonts,amsthm,amssymb, amsaddr}


\title{Maxwell's Equations}

\author{Joe Bentley}

\date{\today}

\begin{document}

\maketitle

\newpage

\section{Electrostatics}

Consider the electric field $\mathbf{E}(\mathbf{r})$ at a point $\mathbf{r}$ due to a charge distribution $\rho(\mathbf{r'})$. The point $\mathbf{r}$ is the point at which we want to measure the electric field, and the point $\mathbf{r'}$ is the point at which the source of the charge we are considering resides. The infinitesimal charge at that point is given by,

\begin{align*}
  dq' = \rho(\mathbf{r'}) dV'
\end{align*}

The infinitesimal volume $dV'$ can also be written as $dV' = d^3\mathbf{r'}$. Coulomb's law gives us that the infinitesimal electric field at point $\mathbf{r}$ is,

\begin{align*}
  d\mathbf{E}(\mathbf{r}) &= \frac{dq'}{4\pi\epsilon_0 {|\mathbf{r}-\mathbf{r'}|}^2} \frac{(\mathbf{r} - \mathbf{r'})}{|\mathbf{r}-\mathbf{r'}|} \\
  &= \frac{\rho(\mathbf{r'})(\mathbf{r}-\mathbf{r'})}{4\pi\epsilon_0{|\mathbf{r}-\mathbf{r'}|}^3} dV' \\
  \mathbf{E}(\mathbf{r}) &= \frac{\rho(\mathbf{r'})(\mathbf{r}-\mathbf{r'})}{4\pi\epsilon_0{|\mathbf{r}-\mathbf{r'}|}^3} dV'
\end{align*}

In the second line we substitute in our expression for an infinitesimal charge element $dq' = \rho(\mathbf{r'}) dV'$. The divergence of the electric field is then given by,

\begin{align*}
  \nabla\cdot\mathbf{E}(\mathbf{r}) &= \int\frac{\rho(\mathbf{r'})}{4\pi\epsilon_0} \nabla r \cdot \frac{(\mathbf{r}-\mathbf{r'})}{{|\mathbf{r}-\mathbf{r'}|}^3} d^3\mathbf{r'}
  &= \int \frac{\rho(\mathbf{r'})}{4\pi\epsilon_0} 4\pi\delta^3(\mathbf{r} - \mathbf{r'}) d^3\mathbf{r'}
\end{align*}

In the distributions notes we showed that $\nabla\cdot\frac{\mathbf{r}}{r^3}=4\pi\delta^3(\mathbf{r})$, which is what we wrote on the second line of the last equation. By cancelling the values of $4\pi$ and then integrating the delta function over volume, we obtain our first Maxwell equation,

\begin{align*}
  \nabla\cdot\mathbf{E}(\mathbf{r}) = \frac{\rho(\mathbf{r})}{\epsilon_0}
\end{align*}

If we integrate both sides of this over a volume $V$, bounded by a surface $S = \partial V$ and apply the divergence theorem we get,

\begin{align*}
  \int_V \nabla\cdot\mathbf{E} dV = \int_V \frac{\rho(\mathbf{r})}{\epsilon_0} dV 
\end{align*}

Which gives us an equation known as Gauss' law,

\begin{align*}
  \oint_S \mathbf{E}\cdot d\mathbf{S} = \frac{Q}{\epsilon_0}
\end{align*}

We can show that the curl of the electric field is zero, $\nabla\times\mathbf{E}=0$, since we can write the electric field as the gradient of a scalar field $\mathbf{E} = -\nabla\phi$, where $\phi(r)$ is given by,

\begin{align*}
  \phi(r) = \int_V \frac{\rho(\mathbf{r'})d^3\mathbf{r'}}{4\pi\epsilon_0 |\mathbf{r} - \mathbf{r'}|}
\end{align*}

We therefore know that the curl of the electric field must be zero, from the vector identity that the curl of a gradient of a scalar field is always zero $\nabla\times(\nabla\phi) = \mathbf{0}$. However this is only true in the static case, where there is no time dependence in the electric or magnetic fields. We will explore the electromagnetic case later.


\section{Magnetostatics}


The Biot-Savart law gives us that the magnetic field $\mathbf{B}(\mathbf{r})$ due to a current density distribution $\mathbf{j}(\mathbf{r'})$ is given by,
  
\begin{align*}
  \mathbf{B}(\mathbf{r})=\frac{\mu_0}{4\pi}\int_V\frac{\mathbf{j}(\mathbf{r'})\times(\mathbf{r}-\mathbf{r'})}{{|\mathbf{r}-\mathbf{r'}|}^3} d^3\mathbf{r'}
\end{align*}

The cross product shows us that the magnetic field $\mathbf{B}(\mathbf{r})$ will be perpendicular to both the current density $\mathbf{j}(\mathbf{r'})$ (and thus the current) and the radial direction $(\mathbf{r} - \mathbf{r'})$.

We can rewrite this by noting that if a vector $\mathbf{c}$ is constant in space, then we can write,

\begin{align*}
  \nabla\times(\phi\mathbf{c}) = \nabla\phi\times\mathbf{c}+\phi\nabla\times\mathbf{c} = \nabla\phi\times\mathbf{c}
\end{align*}

The second term $\phi\nabla\times\mathbf{c} = 0$ because the curl of a constant vector is zero, since it is unchanging in space so the derivative of each component will be zero. In this case our constant vector is given by,

\begin{align*}
  \mathbf{c} = \frac{(\mathbf{r}-\mathbf{r'})}{{|\mathbf{r}-\mathbf{r'}|}^3}
\end{align*}

and then by applying this to the expression inside the integral in the Biot-Savart law,

\begin{align*}
  \mathbf{j}(\mathbf{r'})\times\frac{(\mathbf{r}-\mathbf{r'})}{{|\mathbf{r}-\mathbf{r'}|}^3} &= -\frac{(\mathbf{r}-\mathbf{r'})}{{|\mathbf{r}-\mathbf{r'}|}^3}\times \\
  &= \left(\nabla_r \frac{1}{|\mathbf{r}-\mathbf{r'}|}\right)\times\mathbf{j}(\mathbf{r'}) \\
  &= \nabla_r\times\left(\frac{\mathbf{j}(\mathbf{r'})}{|\mathbf{r}-\mathbf{r'}|}\right)
\end{align*}

By applying these results, the Biot-Savart law may be rewritten as,

\begin{align*}
  \mathbf{B}(\mathbf{r}) = \frac{\mu_0}{4\pi}\int_V\nabla_r\times\frac{\mathbf{j}(\mathbf{r'})}{|\mathbf{r}-\mathbf{r'}|} d^3\mathbf{r'}
\end{align*}

It follows that we can write the magnetic field as,

\begin{align*}
  \mathbf{B}(\mathbf{r}) = \nabla\times\mathbf{A}(\mathbf{r})
\end{align*}

where,

\begin{align*}
  \mathbf{A}(\mathbf{r}) = \frac{\mu_0}{4\pi}\int_V\frac{\mathbf{j}(\mathbf{r'})}{|\mathbf{r}-\mathbf{r'}|}d^3\mathbf{r'}
\end{align*}

We immediately see from our vector identities, that taking the divergence of the magnetic field is taking the divergence of the curl of a vector field, which our vector identities show is zero, $\nabla\cdot(\nabla\times\mathbf{A})=0$. Therefore the divergence of the magnetic field is zero,

\begin{align*}
  \nabla\cdot\mathbf{B}(\mathbf{r}) = 0
\end{align*}

which is another of Maxwell's equations.

The curl of the magnetic field can be calculated using the vector identity,

\begin{align*}
  \nabla\times\mathbf{B}=\nabla\times(\nabla\times\mathbf{A})=\nabla(\nabla\cdot\mathbf{A})-\nabla^2\mathbf{A}
\end{align*}

First we will calculate the laplacian of the vector field, as this is much easier than the first term,

\begin{align*}
  \nabla_r^2\mathbf{A}(\mathbf{r})&=\frac{\mu_0}{4\pi}\int_V\left(\nabla_r^2\frac{1}{|\mathbf{r}-\mathbf{r'}|}\right)\mathbf{j}(\mathbf{r'})d^3\mathbf{r'} \\
  &= \frac{\mu_0}{4\pi}\int_V -4\pi\delta^3(\mathbf{r}-\mathbf{r'})\mathbf{j}(\mathbf{r'})d^3\mathbf{r'} \\
  &= -\mu_0\mathbf{j}(\mathbf{r'})
\end{align*}

Next we need to calculate the first term. First we will try and work out a nice form for the divergence of $\mathbf{A}$,

\begin{align*}
  \nabla_r\cdot\int_V\frac{\mathbf{j}(\mathbf{r'})}{|\mathbf{r}-\mathbf{r'}|}d^3\mathbf{r'}=\int_V\nabla_r\cdot\left[\frac{\mathbf{j}(\mathbf{r'})}{|\mathbf{r}-\mathbf{r'}|}\right]d^3\mathbf{r'}
\end{align*}

Much as with the curl of a scalar field multiplied by a constant vector, we can write $\nabla\cdot(\phi\mathbf{c})=(\nabla\phi)\cdot\mathbf{c}+\phi\nabla\cdot\mathbf{c}=(\nabla\phi)\cdot\mathbf{c}$. We know that the second term is zero, since the divergence of a constant vector must be zero due to the fact that vector is unchanging in space. We can therefore write,

\begin{align*}
  \int_V\nabla_r\left[\frac{\mathbf{j}(\mathbf{r'})}{|\mathbf{r}-\mathbf{r'}|}\right]d^3\mathbf{r'}=\int_V\left[\nabla_r\frac{1}{|\mathbf{r}-\mathbf{r'}|}\right]\cdot\mathbf{j}(\mathbf{r'})d^3\mathbf{r'}
\end{align*}

We want to know this in terms of $\nabla_{r'}$ instead of $\nabla_r$. We can write,

\begin{align*}
  \nabla_{r'} \frac{1}{|\mathbf{r}-\mathbf{r'}|} = \frac{(\mathbf{r}-\mathbf{r'})}{{|\mathbf{r}-\mathbf{r'}|}^3} = -\nabla_r\frac{1}{|\mathbf{r}-\mathbf{r'}|}
\end{align*}

Using this we can write,

\begin{align*}
  \nabla_r\cdot\int_v\frac{\mathbf{j}(\mathbf{r'})}{|\mathbf{r}-\mathbf{r'}|}d^3\mathbf{r}=-\int_V\left[\nabla_{r'}\frac{1}{|\mathbf{r}-\mathbf{r'}|}\right]\cdot\mathbf{j}(\mathbf{r'})d^3\mathbf{r'}
\end{align*}

Here we can use the vector identity, $\nabla\cdot(\phi\mathbf{A})=(\nabla\phi)\cdot\mathbf{A}+\phi\nabla\cdot\mathbf{A}$, so that we can write,

\begin{align*}
  \nabla_{r'}\cdot\left[\frac{\mathbf{j}(\mathbf{r'})}{|\mathbf{r}-\mathbf{r'}|}\right] = \left[\nabla_{r'}\frac{1}{|\mathbf{r}-\mathbf{r'}|}\right]\cdot\mathbf{j}(\mathbf{r'}) + \frac{1}{|\mathbf{r}-\mathbf{r'}|}\nabla_{r'}\cdot\mathbf{j}(\mathbf{r'})
\end{align*}

Subtituting this back into our integral we obtain,

\begin{align*}
  \nabla_r\cdot\int\frac{\mathbf{j}(\mathbf{r'})}{|\mathbf{r}-\mathbf{r'}|}d^3\mathbf{r'}&=-\int_V\nabla_{r'}\cdot\left[\frac{\mathbf{j}(\mathbf{r'})}{|\mathbf{r}-\mathbf{r'}|}\right]d^3\mathbf{r'} + \int_V\frac{\nabla_{r'}\cdot\mathbf{j}(\mathbf{r'})}{|\mathbf{r}-\mathbf{r'}|}d^3\mathbf{r'} \\
  &=-\oint_S\frac{\mathbf{j}(\mathbf{r'})\cdot d\mathbf{S'}}{|\mathbf{r}-\mathbf{r'}|} + \int_V\frac{\nabla_{r'}\cdot\mathbf{j}(\mathbf{r'})}{|\mathbf{r}-\mathbf{r'}|}d^3\mathbf{r'}
\end{align*}


For the first term we have used the divergence theorem to convert the volume integral into a surface intagral at infinity. Since $\mathbf{j}(\mathbf{r'})=\mathbf{0}$ along the surface, this integral will be zero.

From the continuity equation we have that,

\begin{align}
  \label{eq:continuity}
  \nabla_{r'}\cdot\mathbf{j}(\mathbf{r'}, t) = -\frac{\partial\rho(\mathbf{r'}, t)}{\partial t} = 0
\end{align}

This is zero as charge (and thus current) is always conserved.

Since we have already shown that the first integral is zero, and this shows that the second integral is zero, we therefore have that in the static case,

\begin{align*}
  \nabla_{r'}\cdot\mathbf{A}(\mathbf{r}) = \frac{\mu_0}{4\pi}\int_V\nabla_r\cdot\left[\frac{\mathbf{j}(\mathbf{r'})}{|\mathbf{r}-\mathbf{r'}|}\right]d^3\mathbf{r'} = 0
\end{align*}

Therefore, in the case where the magnetic and electric fields are constant in time, we have that,

\begin{align*}
  \nabla\times\mathbf{B}(\mathbf{r})&=\nabla(\nabla\cdot\mathbf{A})-\nabla^2\mathbf{A} \\
  &=\mu_0\mathbf{j}(\mathbf{r})
\end{align*}

In the general, non static case, we calculate the divergence of $\mathbf{A}$ as,

\begin{align*}
  \nabla_r\cdot\mathbf{A}(\mathbf{r}, t)&=\frac{\mu_0}{4\pi}\int_V\nabla_r\cdot\left[\frac{\mathbf{j}(\mathbf{r'}, t)}{|\mathbf{r}-\mathbf{r'}|}\right]d^3\mathbf{r'} \\
  &=\frac{\mu_0}{4\pi}\int_V\frac{\nabla_{r'}\cdot\mathbf{j}(\mathbf{r'}, t)}{|\mathbf{r}-\mathbf{r'}|}d^3\mathbf{r'} \\
  &=-\frac{\mu_0}{4\pi}\frac{\partial}{\partial t}\int_V\frac{\rho(\mathbf{r'}, t)}{|\mathbf{r}-\mathbf{r'}|}d^3\mathbf{r'}
\end{align*}

In the third line we have used the continuity equation, eq.~\ref{eq:continuity}. By taking the gradient of this we get the result we need,

\begin{align*}
  \nabla_r\left(\nabla_r\cdot\mathbf{A}(\mathbf{r}, t)\right) &=\frac{\mu_0}{4\pi}\frac{\partial}{\partial t}\int_{V'}\frac{\rho(\mathbf{r'}, t)}{{|\mathbf{r}-\mathbf{r'}|}^3}d^3\mathbf{r'} \\
  &= \mu_0\epsilon_0\frac{\partial\mathbf{E}(\mathbf{r}, t)}{\partial t}
\end{align*}

We therefore have, in the general case where the magnetic and electric field need not be constant,

\begin{align*}
  \nabla\times\mathbf{B}=\mu_0\mathbf{j}+\mu_0\epsilon_0\frac{\partial\mathbf{E}}{\partial t}
\end{align*}











\end{document}
