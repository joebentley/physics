\documentclass[11pt]{amsart}
\usepackage{amsmath,amsfonts,amsthm,amssymb, amsaddr}


\title{Fourier Series}

\author{Joe Bentley}

\date{\today}

\begin{document}

\maketitle

\newpage

\section{Periodicity}

A function $f(x)$ is said to be periodic with period $L$ if it has the same value at $x$ and $x + L$,

\begin{align*}
  f(x + L) = f(x)
\end{align*}

If a function has a period $L$ then it clearly also has period $nL$ for any positive integer $n$, as shown by,

\begin{align*}
  f(x + 2L) = f\left((x+L)+L\right) = f(x+L) = f(x)
\end{align*}

The minimum period of a periodic function, which cannot be divided into any more sections, is called the fundamental period of the function. For example the fundamental period of $f(x) = \sin{x}$ is $2\pi$.

If we know the value of a periodic function $f(x)$ over an interval of the length of its fundamental period $L$, we know the value of the function everywhere, as it is just that interval repeated.

\section{Fourier Series}

Sine and cosine are periodic functions with a fundamental period of $2\pi$. These functions are useful as we know much about them, for example we can integrate and differentiate them, and we can evaluate them for any value of $x$. Therefore it would be useful to be able to use them to represent a periodic function, much as we do with the maclaurin series except that instead of using a polynomial, we are using sines and cosines.

$\sin nx$ and $\cos nx$ are also periodic over $2\pi$, as we have shown earlier, as long as $n$ is a positive integer. It follows that $\sin{\frac{2\pi nx}{L}}$ and $\cos{\frac{2\pi nx}{L}}$ are periodic over length $L$. We can see this because if $x = L$, then it cancels to give $\sin{2\pi n}$, which we know represents a full cycle as the period of $\sin{nx}$ is $2\pi$. Think of this like the wavenumber, $k = \frac{2\pi}{\lambda}$.

As we have mentioned, the idea of the Fourier series is to be able to write function $f(x)$ of period $L$ in terms of sines and cosines of the same period $L$ but different amplitudes.

\begin{align*}
  f(x) = \frac{1}{2}a_0 + \sum\limits_{n=1}^{\infty}\left[a_n\cos{\frac{2\pi nx}{L}} + b_n\sin{\frac{2\pi nx}{L}}\right]
\end{align*}

We will show that we can find these coefficients for a function $f(x)$ by evaluating the overlap integral over a full period $L$,

\begin{align*}
  a_n &= \frac{2}{L}\int_0^L f(x) \cos{\frac{2\pi nx}{L}} dx \\
  b_n &= \frac{2}{L}\int_0^L f(x) \sin{\frac{2\pi nx}{L}} dx
\end{align*}

Note that the limits do not have to be from $0$ to $L$, but just need to be over any length of the fundamental period, so for example can be from $-L/2$ to $L/2$.


\section{Orthogonality Relations}

In this section we will prove that the functions $\sin{\frac{2\pi nx}{L}}$ (where $n = 1, 2, 3\dots$) and $\cos{\frac{2\pi nx}{L}}$ (where $n = 0, 1, 2\dots$) can be considered mutually orthogonal, such that if $\phi(x)$ and $\psi(x)$ are two different functions from the set then,

\begin{align*}
  \int_0^L\phi(x)\psi(x) dx = 0
\end{align*}

To show this we will consider the three cases where, $\phi = \cos$ and $\psi = \cos$, $\phi = \sin$ and $\psi = \sin$, and $\phi = \sin$ and $\psi = \cos$.

First consider the integral,

\begin{align*}
  I_{mn} = \int_0^L \cos{\frac{2\pi mx}{L}} \cos{\frac{2\pi nx}{L}} dx
\end{align*}

For this we will use a trigonometric product identity (which can be obtained from the addition formula),

\begin{align*}
  \cos{A}\cos{B} = \frac{1}{2}\left[\cos(A+B)+\cos(A-B)\right]
\end{align*}

Plugging this into our integral,

\begin{align*}
  I_{mn} &= \frac{1}{2}\int_0^L\left[\cos{\frac{2\pi(m+n)x}{L}}+\cos{\frac{2\pi(m-n)x}{L}}\right] dx \\
         &= \frac{L}{4\pi}\left[\frac{1}{m+n}\sin{\frac{2\pi(m+n)x}{L}}+\frac{1}{m-n}\sin{\frac{2\pi(m-n)x}{L}}\right]_0^L
\end{align*}

This has different values for different $m$ and $n$. For $m \neq n$, $I_{mn} = 0$, since both of the sine functions are periodic over the limits of $L$. For $m = n \neq 0$, $I_{mn} = \frac{1}{2} L$, and finally for $m = n = 0$, $I_{mn} = L$.

Next we need to evaulate the integral,

\begin{align*}
  J_{mn} = \int_0^L \sin{\frac{2\pi mx}{L}} \sin{\frac{2\pi nx}{L}} dx
\end{align*}

This time we need the product identity for sine,

\begin{align*}
  \sin{A}\sin{B} = \frac{1}{2}\left[\cos(A-B)-\cos(A+B)\right]
\end{align*}

The result is simply the negative of what we had before,

\begin{align*}
  J_{mn} = \frac{1}{2}\int_0^L\left[\cos{\frac{2\pi(m-n)x}{L}}-\cos{\frac{2\pi(m+n)x}{L}}\right] dx
\end{align*}

This time the results are the same unless $m = n = 0$. We have for $m \neq n$, that $J_{mn} = 0$. For $m = n$ we have $J_{mn} = \frac{1}{2} L$ (even if $m = n = 0$).

Finally we need to consider the integral for sine and cosine. This time we need the product identity $\sin{A}\cos{B} = \frac{1}{2}\left[\sin(A+B)+\sin(A-B)\right]$,

\begin{align*}
  K_{mn} &= \int_0^L\sin{\frac{2\pi mx}{L}}\cos{\frac{2\pi nx}{L}} dx \\
         &= \frac{1}{2}\int_0^L\left[\sin{\frac{2\pi(m+n)x}{L}}+\sin{\frac{2\pi(m-n)x}{L}}\right] dx \\
         &= \frac{L}{4\pi}\left[-\frac{1}{m+n}\cos{\frac{2\pi(m+n)x}{L}}-\frac{1}{m-n}\cos{\frac{2\pi(m-n)x}{L}}\right]_0^L \\
         &= 0
\end{align*}

Again, by periodicity, $K_{mn} = 0$, for all $m$ and $n$.

We have shown therefore that the sine and cosine functions are orthogonal. This is similar to the scalar product, for example let's say we have a vector $\mathbf{A} = A_x\hat{\imath}+A_y\hat{\jmath}+A_z\hat{k}$, if we want the $x$ component of the vector we can use the scalar product $\hat{\imath}\cdot\mathbf{A} = A_x$. We can use this similarly to find the coefficients in the fourier series, but instead using an integral.

If we take the formula for the Fourier series and multiply both sides by $\cos{\frac{2\pi nx}{L}}$ and then integrate,

\begin{align*}
  \int_0^L f(x)\cos{\frac{2\pi nx}{L}} dx &= \frac{1}{2} a_0 \int_0^L\cos{\frac{2\pi nx}{L}} dx \\
                                          &+ \sum\limits_{m=1}^{\infty}\left[a_m\int_0^L\cos{\frac{2\pi mx}{L}}\cos{\frac{2\pi nx}{L}} dx + b_m\int_0^L\sin{\frac{2\pi mx}{L}}\cos{\frac{2\pi nx}{L}} dx\right]
\end{align*}

We know that the first term goes to zero due to periodicity. We also know that the last term inside the sum will be zero for all $m$ and $n$. Finally we know that the first term in the sum will be zero if $m \neq n$, but if $m = n \neq 0$ it will be $\frac{1}{2}L$, so we can represent this using the Kronecker delta $\delta_mn$. Now the integral goes to,

\begin{align*}
  \int_0^L f(x)\cos{\frac{2\pi nx}{L}} dx &= \sum_{m=1}^{\infty} a_m\frac{1}{2}\delta{mn} \\
                                          &= \frac{1}{2}L a_n
\end{align*}

Therefore we have shown that the coefficient $a_n$ can be written in the form,

\begin{align*}
  a_n = \frac{2}{L}\int_0^L f(x)\cos{\frac{2\pi nx}{L}} dx
\end{align*}

for $n = 1, 2, 3\dots$.

If we instead multiply by $\sin{\frac{2\pi nx}{L}}$ and integrate, we get the same result,

\begin{align*}
  b_n = \frac{2}{L}\int_0^L f(x)\sin{\frac{2\pi nx}{L}} dx
\end{align*}

for $n = 0, 1, 2\dots$.

To find the coefficient $a_0$, we take the formula for the Fourier series and integrate it alone,

\begin{align*}
  \int_0^L f(x)dx = \frac{1}{2} a_0 \int_0^L dx = \frac{1}{2} L a_0
\end{align*}

Therefore,

\begin{align*}
  a_0 = \frac{2}{L}\int_0^L f(x) dx
\end{align*}


\section{Finding the Fourier Series}

In the following sections we will apply what we know to a few functions and evaluate their Fourier series.

First let's consider what happens if our function $f(x)$, which we wish to evaluate the Fourier series for, is even such that $f(x) = f(-x)$. In this case,

\begin{align*}
  b_n = \frac{2}{L}\int_{-\frac{L}{2}}^{\frac{L}{2}}f(x)\sin{\frac{2\pi nx}{L}}dx = 0
\end{align*}

so only $a_n$ terms remain.

Similarly, if $f(x)$ is odd such that $f(x) = -f(-x)$,

\begin{align*}
  a_n = \frac{2}{L}\int_{-\frac{L}{2}}^{\frac{L}{2}}f(x)\cos{\frac{2\pi nx}{L}}dx = 0
\end{align*}

so only $b_n$ terms remain.

\section{Finding the Fourier Series: Example 1}

Consider the periodic function which is defined over the interval $x\in\left(-\frac{L}{2}, \frac{L}{2}\right)$ as $f(x) = |x|$. This can be written equivalently as,

\begin{align*}
  f(x)=
  \begin{cases}
    -x -\frac{L}{2} \leq & x \leq 0 \\
    x 0 \leq & x \leq \frac{L}{2} \\
  \end{cases}
\end{align*}

This function is clearly even, as $f(x) = f(-x)$, so we only get cosine ($a_n$) coefficients,

\begin{align*}
  a_n = \frac{2}{L}\int_{-\frac{L}{2}}^{\frac{L}{2}} |x| \cos{\frac{2\pi nx}{L}} dx
\end{align*}

Since the function is even, this means that it is symmetric over the limits, so we can just integrate over half the limits (for example from $0$ to $L/2$ instead of $-L/2$ to $L/2$) and then multiply by two,

\begin{align*}
  a_n = \frac{4}{L}\int_0^{\frac{L}{2}} x\cos{\frac{2\pi nx}{L}} dx
\end{align*}

To simplify solving this we make the solution $y = \frac{2\pi nx}{L}$ and thus $x = \frac{L}{2\pi n} y$ and $dx = \frac{L}{2\pi n}dy$. The integral therefore becomes,

\begin{align*}
  a_n &= \frac{4}{L}{\left(\frac{L}{2\pi n}\right)}^2\int_0^{\pi n} y\cos{y} dy \\
      &= \frac{L}{\pi^2n^2}{\left[y\sin{y}+\cos{y}\right]}_0^{\pi n} \\
      &= \frac{L}{\pi^2n^2}\left[\cos{\pi n} - 1\right] \\
      &= \frac{L}{\pi^2n^2}\left[{(-1)}^n - 1\right]
\end{align*}

In the third line we noted that $\cos{\pi n}$ will be negative for odd integers of $n$, and positive for even integers of $n$. If it is positive then it cancels out with the other $1$ to give zero in brackets. If it is negative it will add up to give two inside the brackets. This behaviour is summarized here,

\begin{align*}
  a_n=
  \begin{cases}
    -\frac{2L}{\pi^2n^2} & \text{n is odd} \\
    0 & \text{n is even}
  \end{cases}
\end{align*}

Finally we need to calculate our zeroth coefficient, $a_0$,

\begin{align*}
  a_0 = \frac{4}{L}\int_0^{\frac{L}{2}} x dx = \frac{4}{L}{\left[\frac{1}{2}x^2\right]}_0^{\frac{L}{2}} = \frac{L}{2}
\end{align*}

We now have all the information we need to subsitute into the Fourier series,

\begin{align*}
  f(x) &= \frac{L}{4}-\frac{2L}{\pi^2}\sum\limits_{\text{n odd}}\frac{1}{n^2}\cos{\frac{2\pi nx}{L}} \\
       &= \frac{L}{4}-\frac{2L}{\pi^2}\sum\limits_{k=0}^{\infty}\frac{1}{{(2k+1)}^2}\cos{\frac{2\pi(2k+1)x}{L}}
\end{align*}

By letting $x = 0$ we can find the value of the sum of $\frac{1}{{(2k+1)}^2}$,

\begin{align*}
  \sum\limits_{k=0}^{\infty} \frac{1}{{(2k+1)}^2} = \frac{\pi^2}{8}
\end{align*}

\section{The Riemann Zeta Function}

In this section we will explore the Riemann Zeta function, but this is not examinable material. The Riemann Zeta function is defined as,

\begin{align*}
  \zeta(z) = \sum\limits_{n=1}^{\infty} \frac{1}{n^z} \qquad \Re{z} > 1
\end{align*}

For other values of $z$ the value of $\zeta{z}$ can be found by using analytic continuoations. From the definition we see that,

\begin{alignat*}{8}
  \zeta(2) &= 1 + &&\frac{1}{2^2} &&+ \frac{1}{3^2} &&+ \frac{1}{4^2} &&+ \frac{1}{5^2} &&+ \frac{1}{6^2} &&+ \frac{1}{7^2} &&+ \dots \\
  \frac{1}{4}\zeta(2) &= &&\frac{1}{2^2} && &&+ \frac{1}{4^2} && &&+ \frac{1}{5^2} && &&+ \dots \\
  \zeta(2) - \frac{1}{4}\zeta(2) = \frac{3}{4}\zeta(2) &= 1 && &&+ \frac{1}{3^2} && &&+ \frac{1}{5^2} && &&+ \frac{1}{7^2} &&+ \dots \\
\end{alignat*}

We can see that the last sequence is just equal to the sum that we found at the end of the last section, as it is just hte sum of all odd $n$ squared. Therefore $\frac{3}{4}\zeta(2) = \frac{\pi^2}{8}$, and it follows that,

\begin{align*}
  \zeta(2) = 1 + \frac{1}{4} + \frac{1}{9} + \frac{1}{16} + \frac{1}{25} + \dots = \frac{\pi^2}{6}
\end{align*}

$\zeta(2n)$ where $n$ is a positive integer, is always a rational multiple of $\pi^{2n}$. The general formula is given by,

\begin{align*}
  \zeta(2n) = \frac{2^{2n-1} |B_{2n}|}{2n!} \pi^{2n}
\end{align*}

where $B_k$ are the Bernoulli numbers defined as,

\begin{align*}
  \frac{z}{e_z - 1} = \sum\limits_{k=0}^{\infty}\frac{B_k}{k!} z^k
\end{align*}

The only non-zero values of $B_k$ for odd $k$ is $B_1$, all others are odd. The first few Bernoulli numbers are,

\begin{align*}
  B_0 = 1 \qquad B_1 = -\frac{1}{2} \qquad B_2 = \frac{1}{6} \qquad B_4 = -\frac{1}{30} \qquad B_6 = \frac{1}{42}
\end{align*}


\section{Parseval's Theorem}

In this section we return to examinable stuff. Consider the Fourier series of a periodic function $f(x)$ given by,

\begin{align*}
  f(x) = \frac{1}{2}a_0 + \sum\limits_{n=1}^{\infty}\left[a_n\cos{\frac{2\pi nx}{L}} + b_n\sin{\frac{2\pi nx}{L}}\right]
\end{align*}

If we square $f(x)$ we get a complicated double series. It we then take the integral over a full period length $L$ then orthogonality means that we only get the diagonal (squared) terms, as we have shown that the integral over a full period of a product of two different cosines, two different sines, or a sine and a cosine is zero,

\begin{align*}
  \int_0^L f^2(x) dx = \frac{1}{4}a_0^2\int_0^Ldx + \sum\limits_{n=1}^{\infty}\left[\int_0^L a_n^2\cos^2{\frac{2\pi nx}{L}} dx + \int_0^L b_n^2 \sin^2{\frac{2\pi nx}{L}} dx\right]
\end{align*}

Or since $\sin^2{x} + \cos^2{x} = 1$ this can be written as,

\begin{align*}
  \int_0^L f^2(x) dx = \frac{1}{2}L\left[\frac{1}{2}a_0^2 + \sum\limits_{n=1}^{\infty}\left(a_n^2+b_n^2\right)\right]
\end{align*}

Note that the limits on the integral can be over any full period length, and don't have to be between $0$ and $L$.

\end{document}
