\documentclass[11pt]{amsart}
\usepackage{amsmath,amsfonts,amsthm,amssymb, amsaddr}


\title{Fourier Series}

\author{Joe Bentley}

\date{\today}

\begin{document}

\maketitle

\newpage

\section{Periodicity}

A function $f(x)$ is said to be periodic with period $L$ if it has the same value at $x$ and $x + L$,

\begin{align*}
  f(x + L) = f(x)
\end{align*}

If a function has a period $L$ then it clearly also has period $nL$ for any positive integer $n$, as shown by,

\begin{align*}
  f(x + 2L) = f\left((x+L)+L\right) = f(x+L) = f(x)
\end{align*}

The minimum period of a periodic function, which cannot be divided into any more sections, is called the fundamental period of the function. For example the fundamental period of $f(x) = \sin{x}$ is $2\pi$.

If we know the value of a periodic function $f(x)$ over an interval of the length of its fundamental period $L$, we know the value of the function everywhere, as it is just that interval repeated.

\section{Fourier Series}

Sine and cosine are periodic functions with a fundamental period of $2\pi$. These functions are useful as we know much about them, for example we can integrate and differentiate them, and we can evaluate them for any value of $x$. Therefore it would be useful to be able to use them to represent a periodic function, much as we do with the maclaurin series except that instead of using a polynomial, we are using sines and cosines.

$\sin nx$ and $\cos nx$ are also periodic over $2\pi$, as we have shown earlier, as long as $n$ is a positive integer. It follows that $\sin{\frac{2\pi nx}{L}}$ and $\cos{\frac{2\pi nx}{L}}$ are periodic over length $L$. We can see this because if $x = L$, then it cancels to give $\sin{2\pi n}$, which we know represents a full cycle as the period of $\sin{nx}$ is $2\pi$. Think of this like the wavenumber, $k = \frac{2\pi}{\lambda}$.

As we have mentioned, the idea of the Fourier series is to be able to write function $f(x)$ of period $L$ in terms of sines and cosines of the same period $L$ but different amplitudes.

\begin{align*}
  f(x) = \frac{1}{2}a_0 + \sum\limits_{n=1}^{\infty}\left[a_n\cos{\frac{2\pi nx}{L}} + b_n\sin{\frac{2\pi nx}{L}}\right]
\end{align*}

We will show that we can find these coefficients for a function $f(x)$ by evaluating the overlap integral over a full period $L$,

\begin{align*}
  a_n &= \frac{2}{L}\int_0^L f(x) \cos{\frac{2\pi nx}{L}} dx \\
  b_n &= \frac{2}{L}\int_0^L f(x) \sin{\frac{2\pi nx}{L}} dx
\end{align*}

Note that the limits do not have to be from $0$ to $L$, but just need to be over any length of the fundamental period, so for example can be from $-L/2$ to $L/2$.


\section{Orthogonality Relations}

In this section we will prove that the functions $\sin{\frac{2\pi nx}{L}}$ (where $n = 1, 2, 3\dots$) and $\cos{\frac{2\pi nx}{L}}$ (where $n = 0, 1, 2\dots$) can be considered mutually orthogonal, such that if $\phi(x)$ and $\psi(x)$ are two different functions from the set then,

\begin{align*}
  \int_0^L\phi(x)\psi(x) dx = 0
\end{align*}

To show this we will consider the three cases where, $\phi = \cos$ and $\psi = \cos$, $\phi = \sin$ and $\psi = \sin$, and $\phi = \sin$ and $\psi = \cos$.

First consider the integral,

\begin{align*}
  I_{mn} = \int_0^L \cos{\frac{2\pi mx}{L}} \cos{\frac{2\pi nx}{L}} dx
\end{align*}

For this we will use a trigonometric product identity (which can be obtained from the addition formula),

\begin{align*}
  \cos{A}\cos{B} = \frac{1}{2}\left[\cos(A+B)+\cos(A-B)\right]
\end{align*}

Plugging this into our integral,

\begin{align*}
  I_{mn} &= \frac{1}{2}\int_0^L\left[\cos{\frac{2\pi(m+n)x}{L}}+\cos{\frac{2\pi(m-n)x}{L}}\right] dx \\
         &= \frac{L}{4\pi}\left[\frac{1}{m+n}\sin{\frac{2\pi(m+n)x}{L}}+\frac{1}{m-n}\sin{\frac{2\pi(m-n)x}{L}}\right]_0^L
\end{align*}

This has different values for different $m$ and $n$. For $m \neq n$, $I_{mn} = 0$, since both of the sine functions are periodic over the limits of $L$. For $m = n \neq 0$, $I_{mn} = \frac{1}{2} L$, and finally for $m = n = 0$, $I_{mn} = L$.

Next we need to evaulate the integral,

\begin{align*}
  J_{mn} = \int_0^L \sin{\frac{2\pi mx}{L}} \sin{\frac{2\pi nx}{L}} dx
\end{align*}

This time we need the product identity for sine,

\begin{align*}
  \sin{A}\sin{B} = \frac{1}{2}\left[\cos(A-B)-\cos(A+B)\right]
\end{align*}

The result is simply the negative of what we had before,

\begin{align*}
  J_{mn} = \frac{1}{2}\int_0^L\left[\cos{\frac{2\pi(m-n)x}{L}}-\cos{\frac{2\pi(m+n)x}{L}}\right] dx
\end{align*}

This time the results are the same unless $m = n = 0$. We have for $m \neq n$, that $J_{mn} = 0$. For $m = n$ we have $J_{mn} = \frac{1}{2} L$ (even if $m = n = 0$).

Finally we need to consider the integral for sine and cosine. This time we need the product identity $\sin{A}\cos{B} = \frac{1}{2}\left[\sin(A+B)+\sin(A-B)\right]$,

\begin{align*}
  K_{mn} &= \int_0^L\sin{\frac{2\pi mx}{L}}\cos{\frac{2\pi nx}{L}} dx \\
         &= \frac{1}{2}\int_0^L\left[\sin{\frac{2\pi(m+n)x}{L}}+\sin{\frac{2\pi(m-n)x}{L}}\right] dx \\
         &= \frac{L}{4\pi}\left[-\frac{1}{m+n}\cos{\frac{2\pi(m+n)x}{L}}-\frac{1}{m-n}\cos{\frac{2\pi(m-n)x}{L}}\right]_0^L \\
         &= 0
\end{align*}

Again, by periodicity, $K_{mn} = 0$, for all $m$ and $n$.

We have shown therefore that the sine and cosine functions are orthogonal. This is similar to the scalar product, for example let's say we have a vector $\mathbf{A} = A_x\hat{\imath}+A_y\hat{\jmath}+A_z\hat{k}$, if we want the $x$ component of the vector we can use the scalar product $\hat{\imath}\cdot\mathbf{A} = A_x$. We can use this similarly to find the coefficients in the fourier series, but instead using an integral.

If we take the formula for the Fourier series and multiply both sides by $\cos{\frac{2\pi nx}{L}}$ and then integrate,

\begin{align*}
  \int_0^L f(x)\cos{\frac{2\pi nx}{L}} dx &= \frac{1}{2} a_0 \int_0^L\cos{\frac{2\pi nx}{L}} dx \\
                                          &+ \sum\limits_{m=1}^{\infty}\left[a_m\int_0^L\cos{\frac{2\pi mx}{L}}\cos{\frac{2\pi nx}{L}} dx + b_m\int_0^L\sin{\frac{2\pi mx}{L}}\cos{\frac{2\pi nx}{L}} dx\right]
\end{align*}

We know that the first term goes to zero due to periodicity. We also know that the last term inside the sum will be zero for all $m$ and $n$. Finally we know that the first term in the sum will be zero if $m \neq n$, but if $m = n \neq 0$ it will be $\frac{1}{2}L$, so we can represent this using the Kronecker delta $\delta_mn$. Now the integral goes to,

\begin{align*}
  \int_0^L f(x)\cos{\frac{2\pi nx}{L}} dx &= \sum_{m=1}^{\infty} a_m\frac{1}{2}\delta{mn} \\
                                          &= \frac{1}{2}L a_n
\end{align*}

Therefore we have shown that the coefficient $a_n$ can be written in the form,

\begin{align*}
  a_n = \frac{2}{L}\int_0^L f(x)\cos{\frac{2\pi nx}{L}} dx
\end{align*}

for $n = 1, 2, 3\dots$.

If we instead multiply by $\sin{\frac{2\pi nx}{L}}$ and integrate, we get the same result,

\begin{align*}
  b_n = \frac{2}{L}\int_0^L f(x)\sin{\frac{2\pi nx}{L}} dx
\end{align*}

for $n = 0, 1, 2\dots$.

To find the coefficient $a_0$, we take the formula for the Fourier series and integrate it alone,

\begin{align*}
  \int_0^L f(x)dx = \frac{1}{2} a_0 \int_0^L dx = \frac{1}{2} L a_0
\end{align*}

Therefore,

\begin{align*}
  a_0 = \frac{2}{L}\int_0^L f(x) dx
\end{align*}


\end{document}
