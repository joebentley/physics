\documentclass[11pt]{amsart}
\usepackage{amsmath,amsfonts,amsthm,amssymb, amsaddr}


\title{Distributions (Generalised Functions)}

\author{Joe Bentley}

\date{\today}

\begin{document}

\maketitle

\newpage

\section{Dirac Delta Function}

Consider a vector function defined as $\mathbf{A} = \frac{\mathbf{r}}{r^3}$. In spherical polar coordinates this can be written as $\mathbf{A} = \frac{1}{r^2} \mathbf{e_r}$. Next we calculate its divergence using the formula proved in previous notes,

\begin{align*}
  \nabla\cdot\mathbf{A} = \frac{1}{r^2}\frac{\partial}{\partial r}\left(r^2 \frac{1}{r^2}\right) = 0
\end{align*}

Consider next the divergence theorem for the same vector field $\mathbf{A}$, also known as Gauss' theorem, which was shown to be,

\begin{align*}
  \int_V \nabla\cdot\mathbf{A} dV = \oint_S \mathbf{A}\cdot d\mathbf{S}
\end{align*}

By integrating over a sphere of radius $a$ centered at the origin we see,

\begin{align*}
  \oint_S \mathbf{A}\cdot d\mathbf{S} = \frac{1}{a^2} 4\pi a^2 = 4\pi
\end{align*}

There appears to be a contradiction as we have just proved that $\nabla\cdot\mathbf{A} = 0$, but this integral tells us that since the surface integral is non-zero, then so is the volume integral. It turns out however that we have only really proved that $\nabla\cdot\mathbf{A} = 0$ when $\mathbf{r} \neq \mathbf{0}$. Therefore we must have that $\nabla\cdot\mathbf{A}$ suddenly blows up at $\mathbf{r} = \mathbf{0}$ in such a way that the integral over it has the value $4\pi$. This result is given by a function called the Dirac delta function, in this case in 3 dimensions,

\begin{align*}
  \nabla\cdot\mathbf{A} = 4\pi \delta^3(\mathbf{r})
\end{align*}

where

\begin{align*}
  \delta^3(\mathbf{r}) =
  \begin{cases}
    0 & \mathbf{r} \neq \mathbf{0} \\
    \infty & \mathbf{r} = \mathbf{0}
  \end{cases}
\end{align*}

which also has the constraint,

\begin{align*}
  \int_V \delta^3(\mathbf{r}) d^3\mathbf{r} = 1
\end{align*}

For example, the electric field due to a point charge at the origin is given by,

\begin{align*}
  \mathbf{E} = \frac{q}{4\pi\epsilon_0} \frac{\mathbf{r}}{r^3}
\end{align*}

The divergence of this can thus be written as,

\begin{align*}
  \nabla\cdot\mathbf{E} = \frac{q}{\epsilon_0}\delta^3(\mathbf{r}) = \frac{\rho(\mathbf{r})}{\epsilon_0}
\end{align*}

Here, $\rho(\mathbf{r}) = q\delta^3(\mathbf{r})$ is the charge density of a point charge $q$ at the origin. It is zero at $\mathbf{r} \neq \mathbf{0}$ and infinite at $\mathbf{r} = \mathbf{0}$ such that the integral over its volume gives $q$.

The Dirac delta function shows up every time we mix discrete and continuous objects.

\section{The 1D Dirac Delta Function}

The one-dimensional Dirac delta function $\delta(x)$ is defined,

\begin{align*}
  \delta(x) =
  \begin{cases}
    0 & x \neq 0 \\
    \infty & x = 0
  \end{cases}
\end{align*}

Under the constraint,

\begin{align*}
  \int_{-\infty}^{\infty} \delta(x) dx = 1
\end{align*}

As in the three-dimensional case, it may be defined as the limit of a sequence of functions that become increasingly peaked, getting more and more narrow. A good function to choose for this is a Gaussian distribution,

\begin{align*}
  \delta_n(x) = \sqrt{\frac{n}{\pi}} e^{-nx^2}
\end{align*}

We divide by $\sqrt{\frac{n}{\pi}}$ to normalise the distribution so that if we integrate over it we get one regardless of the value of $n$. We can see that the limits give us back the Dirac delta function,

\begin{alignat*}{2}
  &\lim_{n \to \infty} \delta_n(x) = 0 \qquad &&\text{for } x\neq0 \\
  &\lim_{n \to \infty} \delta_n(x) = \infty \qquad &&\text{for } x = 0 \\
  &\lim_{n \to \infty} \int_{-\infty}^{\infty} \delta_n(x) = 1
\end{alignat*}

We can therefore think of the Dirac delta function $\delta(x)$ as the limit of $\delta_n(x)$ as $n \to \infty$.

The key property of the delta function is that,

\begin{align*}
  \int_a^b f(x) \delta(x - x_0) dx =
  \begin{cases}
    f(x_0) & a < x_0 < b \\
    0 & \text{otherwise}
  \end{cases}
\end{align*}

This is because the delta function effectively picks out the value of $f(x)$ where $x = x_0$. This is because the only place where the delta function isn't zero, is when $x = x_0$. Since the integral over the entire Dirac delta function is one, that means we just get the function $f(x_0)$ back.

We may also write this as,

\begin{align*}
  f(x)\delta(x - x_0) = f(x_0)\delta(x - x_0)
\end{align*}

Using some intuition it is clear to see that they are equal, as the product will be zero for all $x \neq x_0$, so we can just replace $f(x)$ with $f(x_0)$

We can relate the one-dimensional and three-dimensional Dirac delta functions by,

\begin{align*}
  \delta^3(\mathbf{r}) = \delta(x)\delta(y)\delta(z)
\end{align*}

Or for a given offset $\mathbf{r_0}$,

\begin{align*}
  \delta^3(\mathbf{r} - \mathbf{r_0}) = \delta(x - x_0)\delta(y - y_0)\delta(z - z_0)
\end{align*}



\end{document}
