\documentclass[11pt]{amsart}
\usepackage{amsmath,amsfonts,amsthm,amssymb, amsaddr}


\title{Distributions}

\author{Joe Bentley}

\date{\today}

\begin{document}

\maketitle

\newpage

\section{Dirac Delta Function}

The Dirac delta function is a distribution that is defined to be zero everywhere except at $x = 0$ where is it is infinite. That is,

\begin{align*}
  \delta(x)=
  \begin{cases}
    0 & x\neq0 \\
    \infty & x=0
  \end{cases}
\end{align*}

It is also constrained to the identity,

\begin{align*}
  \int_{a}^{b} \delta(x) dx = 1 \qquad a < 0 < b
\end{align*}

The Dirac delta function is thus an infinitely tall, infinitely thin spike. We can see that any function integrated over it between $a$ and $b$ is,

\begin{align*}
  \int_{a}^{b} f(x)\delta(x) dx = f(0)
\end{align*}

and therefore,

\begin{align*}
  f(x)\delta(x) = f(0)\delta(x)
\end{align*}

\section{Heaviside Step Function}

The Heaviside step function is defined such that,

\begin{align*}
  \Theta(x) =
  \begin{cases}
    0 & x < 0 \\
    1 & x > 0
  \end{cases}
\end{align*}

The gradient at the step is therefore infinite, but everywhere else it is zero. The gradient is therefore given by the Dirac delta function,

\begin{align*}
  \frac{d\Theta}{dx} = \delta(x)
\end{align*}

\end{document}
